\documentclass{book}

% Packages required to support encoding
\usepackage{ucs}
\usepackage[utf8x]{inputenc}

% Packages required by code


% Packages always used
\usepackage{hyperref}
\usepackage{xspace}
\usepackage[usenames,dvipsnames]{color}
\hypersetup{colorlinks=true,urlcolor=blue}

\title{The Rails Upgrade Handbook}



\begin{document} 

\maketitle



\newpageRails debuted in late 2004, and today, in early 2010, we'{}re looking at a totally different framework in Rails 3.0. The new features, performance improvements, and API changes aren'{}t incredibly drastic, but they do present challenges to those looking to upgrade existing code to the newest version. This e-book is a look at how to go about upgrading your existing Rails app to Rails 3, how to convert common patterns to new ones, and how to improve your existing code in light of the new Rails 3 features. First, though, we should go over some of the high-level philosophical and architectural changes that have gone on in the Rails code between versions 2 and 3.

\hypertarget{the_big_picture}{}\subsection*{{The Big Picture}}\label{the_big_picture}

When the Merb/Rails merge was announced, some members of the Rails community were very interested to see how the final product ended up: was it going to be more of the same, something totally new, or simply Merb 2.0? But as we'{}re approaching a final product, we'{}re finding out that we'{}re getting the best of both worlds: the ease of use and packaging of Rails with the juicy technical bits of Merb. Who can argue with that?

As the team worked together towards a vision for the project, obviously changes were made to the Rails way of doing things. These big picture changes have concentrated on a few key areas:

\begin{itemize}%
\item Decoupling Rails components from one another as much as possible, making things more modular and a la carte.\footnote{http://yehudakatz.com/2009/07/19/rails-3-the-great-decoupling/} 
\item Pulling in improvements from Merb and rewrite/refactor much of the internals to improve performance.\footnote{http://www.engineyard.com/blog/2009/rails-and-merb-merge-performance-part-2-of-6/} 
\item Exposing explicit, documented API'{}s for common tasks and integration of wider ecosystem components from testing, ORM, etc.\footnote{http://www.engineyard.com/blog/2010/rails-and-merb-merge-plugin-api-part-3-of-6} 

\end{itemize}
In order to hit these objectives, DHH, Yehuda, Josh, and the rest of the Rails team have extracted things into some new components, expanded others, and removed others to allow for agnosticism.



The general movement seems to be from a monolithic, one-stop shop approach to a looser ecosystem of code that works together with a straightforward set of sensible defaults. You'{}re no longer ``{}locked in''{} to ActiveRecord or made to use code injection and hacks and such to get your testing framework integrated. Instead, there are hooks all over the place to cover this sort of integration that let generators generate things for the various options or helpers include different modules. It'{}s a great way to support the existing plugin ecosystem, except this time with an established API.

\hypertarget{lifecycle_changes}{}\subsection*{{Lifecycle changes}}\label{lifecycle_changes}

One of the biggest movements in the codebase has been a shift towards using simple, composed components and a lot of Rack features in the request chain rather than specialized, one-off classes. This has affected a lot of things, but one of the major changes has been the addition of \textbf{Action Dispatch}.\footnote{http://github.com/rails/rails/blob/master/actionpack/lib/action\_dispatch.rb} 



Action Dispatch is a ``{}new''{} component in Action Pack (extracted and expanded from the previous logic) that handles a number of things related to requests and responses:

\begin{itemize}%
\item Request handling and parameter parsing
\item Sessions, Rails'{} flash, and cookie storage
\item File uploads
\item Routing, URL matching, and rescuing errors
\item HTTP conditional GETs
\item Client response and HTTP status code

\end{itemize}
Breaking this functionality out into its own component and decoupling much of it creates a much more flexible call stack for requests, meaning you can jack into the process easier with your own logic or improve the existing functionality. I'{}m sure we'{}ll see a lot of plugins taking advantage of this to create interesting middleware hacks, improve callbacks and lifecycle methods, hack in their own middlewares to handle specialized logic, or even plug in improved or application-specific routers.

\hypertarget{making_controllers_flexible}{}\subsection*{{Making controllers flexible}}\label{making_controllers_flexible}

As a result of the changes in the request chain, the controller stack has also seen a significant overhaul. Previously, every controller inherited from {\colorbox[rgb]{0.87,0.87,0.87}{\tt ActionController\char58\char58Base}} (either directly or by inheriting from {\colorbox[rgb]{0.87,0.87,0.87}{\tt ApplicationController}}), and to slim down the call stack, one had to either (a) previous to Rails 2.3, building a smaller app with Sinatra or Rack to sit next to your main Rails application or (b) post-Rails 2.3, using Rails Metal/Rack middlewares.

In Rails 3.0, this concept of middleware plays an even more central role to how the controller hierarchy is arranged.



The bottom of the stack is \href{http://yehudakatz.com/2009/03/20/another-dispatch-abstractcontroller/}{{\colorbox[rgb]{0.87,0.87,0.87}{\tt AbstractController}}}, a very low level ``{}controller.''{} Rails uses this class to abstract away essentials like rendering, layouts, managing template paths, and so on, while leaving more concrete implementation details to its subclasses. {\colorbox[rgb]{0.87,0.87,0.87}{\tt AbstractController}} exists only to provide these facilities to subclasses. That is, you should not use this class directly; if you want something super-slim, create a subclass and implement {\colorbox[rgb]{0.87,0.87,0.87}{\tt render}} and a few other pieces).

Each subsequent jump up the hierarchy is actually a class that inherits from the previous, each including modules to compose its behavior. So, if you want to create something slim without implementing a lot of plumbing, use the next rung on the compositional ladder: \href{http://yardoc.org/docs/rails/ActionController/Metal}{{\colorbox[rgb]{0.87,0.87,0.87}{\tt ActionController\char58\char58Metal}}}. {\colorbox[rgb]{0.87,0.87,0.87}{\tt Metal}} essentially exposes super simple Rack endpoints that you can then include extra modules into to add more {\colorbox[rgb]{0.87,0.87,0.87}{\tt ActionController}} functionality (check out an example \href{http://yehudakatz.com/2009/12/20/generic-actions-in-rails-3/}{here}). These little classes are excellent for replacing those Rack/Sinatra apps for file uploads or what have you while still having the power to easily build out to rather rich controller objects.

Finally, if you need the full monty (i.e., like a controller in Rails 2), then you'{}ll need to inherit from {\colorbox[rgb]{0.87,0.87,0.87}{\tt ActionController\char58\char58Base}}. This class inherits from {\colorbox[rgb]{0.87,0.87,0.87}{\tt ActionController\char58\char58Metal}} and includes a slew of modules to handle things like redirecting the user, handling implicit rendering, and a number of helpers for other stuff like caching.

The advantage of taking this approach is that you can take one of the base classes like {\colorbox[rgb]{0.87,0.87,0.87}{\tt Metal}} and include your own modules to create specialized controllers. I foresee someone using this to create a simple way to serve up resources (e.g., {\colorbox[rgb]{0.87,0.87,0.87}{\tt PostsController~\char60~ResourcesController\char40\char58posts\char41}} or something like that) much like people have done previously or using it as a way to quickly build API backends. This is the other piece of the major refactor that excites me, since we'{}re looking at a new way to construct reusable code and assemble it into usable applications.

\hypertarget{where_models_are_concerned}{}\subsection*{{Where models are concerned}}\label{where_models_are_concerned}

Though the public API for models is generally the same (with a few additions and changes that I'{}ll cover in a subsequent post), Active Record is now powered by the brain-melting Active Relation, a powerful relational algebra layer.



What does that mean for you? Well, basically it means that Active Record will be smarter and more powerful. Rather than fairly naïve SQL generation, it uses some fancy \href{http://db.grussell.org/section010.html}{mathemagical} approach that should generate smarter queries. Frankly, I haven'{}t had a lot of time to research these features for myself, but when I do, I'{}ll be sure to post (or if you'{}ve posted about this stuff somewhere, then by all means let me know).

The second big change in Model Land is the extraction of much of the rich logic in Active Record objects like callbacks, validations, serialization, and so on into the Active Model module.



You can use this module to \href{http://yehudakatz.com/2010/01/10/activemodel-make-any-ruby-object-feel-like-activerecord/}{make any object behave like an Active Record object}; for example, let'{}s say you wanted to add some validations to a PORO representing a host on a network:

\begin{verbatim}class Host
  include ActiveModel::Validations

  validates_presence_of :hostname

  attr_accessor :ip_address, :hostname, :operating_system
  def initialize(hostname, ip_address, operating_system)
    @hostname, @ip_address, @operating_system = host, ip_address, operating_system
  end
end

h  = Host.new("skull", "24.44.129.10", "Linux")
h.valid?    # => true
h.hostname = nil
h.valid?    # => false\end{verbatim}
To get this functionality, simply include {\colorbox[rgb]{0.87,0.87,0.87}{\tt ActiveModel\char58\char58Validations}} and start implementing the methods. It'{}s possible to exercise fine-grained control over how the validations operate, how the validator gets the object'{}s attributes, and so on. To get the other functionality like observing or callbacks, just include the relevant module (e.g., {\colorbox[rgb]{0.87,0.87,0.87}{\tt ActiveModel\char58\char58Observing}}) and implement the required methods. It'{}s fantastically clever.

\hypertarget{other_pieces}{}\subsection*{{Other pieces}}\label{other_pieces}

ActionMailer is also getting some love in Rails 3. A new API pointed out by DHH in \href{http://gist.github.com/281420}{this gist} is looking especially delicious; it'{}s much more like a controller with some excellent helpers mixed in just for mailing.

Rails is also getting a rather robust \href{http://en.wikipedia.org/wiki/Instrumentation_(computer_programming}{instrumentation framework}). In essence, an instrumentation framework lets you subscribe to events inside of a system and respond to them in meaningful ways (e.g., an action renders and the logger logs its result). Internally the framework is used for things like logging and debugging, but you could easily repurpose the code for other things. For example, let'{}s say you want to log to the system logger when a particular e-mail is sent out:

\begin{verbatim}# Subscribe to the event...
ActiveSupport::Notifications.subscribe do |*args|
  @events << ActiveSupport::Notifications::Event.new(*args)
end

# Fire the event...
ActiveSupport::Notifications.instrument(:system_mail, :at => Time.now) do
  #SystemMailer.important_email.deliver
  log "Important system mail sent!"
end

# Do something with it...
event = @events.first
event.name        # => :system_mail
event.payload     # => { :at => Wed Jan 16 00:51:14 -0600 2010 }
event.duration    # => 0.063
system_log(event) # => <whatever>\end{verbatim}
Of course, this is arbitrary, but it adds a really powerful way to respond to certain events in your application. For example, someone could probably rewrite {\colorbox[rgb]{0.87,0.87,0.87}{\tt exception\char95notification}} to use the instrumentation framework to handle and send error e-mails.

Now that we'{}ve looked at some of the core architecture, I'{}d like to shift my focus first to upgrading an application. Originally I had planned on writing about upgrading plugins first, but \href{http://groups.google.com/group/rubyonrails-core/browse_thread/thread/00d8f6f406b96031#}{apparently that API isn'{}t quite stable}. So, I figured rather than write a blog post that will be deprecated in 2 weeks, I'{}d rather write one that will be deprecated in 3-6 months instead. So, this post will focus on getting your app bootable, and it will be followed by a succession of articles that contain tips and scripts to help you upgrade the various components (i.e., routes, models, etc. are topics I'{}m working on right now).

The first step towards an upgraded app you need to take is to actually get Rails 3. As noted in the previous post, you can follow \href{http://yehudakatz.com/2009/12/31/spinning-up-a-new-rails-app/}{Yehuda'{}s directions} or use \href{http://github.com/bry4n/rails3-install}{Bryan Goines'{}s great little script}. Once you'{}ve got it up and running, I suggest you ``{}generate a new app''{} on top of your current one (i.e., run the generator and point the app path to your current Rails 2.x app'{}s path). Running the generator again will actually update the files you need to update, generate the new ones, and so on.

\begin{verbatim}ruby /path/to/rails/railties/bin/rails ~/code/my_rails2_app/\end{verbatim}
Note that the argument is a \emph{path}, not a name as in previous Rails versions. If you got an error about your Ruby version, upgrade it! If you use \href{http://rvm.beginrescueend.com/}{rvm} it'{}ll be totally painless. Now, be careful which files you let Rails replace since a lot of them can be edited much more simply (I'{}ll show you how here) than they can be reconstructed (unless you really like digging around in {\colorbox[rgb]{0.87,0.87,0.87}{\tt git~diff}} and previous revisions), but do take note of what they are since you will likely need to change something in them. As a general list, it'{}s probably safe to let it update these files:

* {\colorbox[rgb]{0.87,0.87,0.87}{\tt Rakefile}} * {\colorbox[rgb]{0.87,0.87,0.87}{\tt README}} * {\colorbox[rgb]{0.87,0.87,0.87}{\tt config\char47boot\char46rb}} * {\colorbox[rgb]{0.87,0.87,0.87}{\tt public\char47\char52\char48\char52\char46html}} (unless you'{}ve customized it) * {\colorbox[rgb]{0.87,0.87,0.87}{\tt public\char47\char53\char48\char48\char46html}} (unless you'{}ve customized it) * {\colorbox[rgb]{0.87,0.87,0.87}{\tt public\char47javascripts\char47\char42}} (if you don'{}t have a lot of version dependent custom JavaScript) * {\colorbox[rgb]{0.87,0.87,0.87}{\tt script\char47\char42}} (they probably wouldn'{}t work with the new Rails 3 stuff in their old form anyhow)

And, you probably don'{}t want to let it update these files since you'{}ve likely made modifications:

* {\colorbox[rgb]{0.87,0.87,0.87}{\tt \char46gitignore}} (unless you don'{}t really care; the new standard one is pretty good) * {\colorbox[rgb]{0.87,0.87,0.87}{\tt app\char47helpers\char47application\char95helper\char46rb}} * {\colorbox[rgb]{0.87,0.87,0.87}{\tt config\char47routes\char46rb}} * {\colorbox[rgb]{0.87,0.87,0.87}{\tt config\char47environment\char46rb}} * {\colorbox[rgb]{0.87,0.87,0.87}{\tt config\char47environments\char47\char42}} (unless you haven'{}t touched these as many people don'{}t) * {\colorbox[rgb]{0.87,0.87,0.87}{\tt config\char47database\char46yml}} * {\colorbox[rgb]{0.87,0.87,0.87}{\tt doc\char47README\char95FOR\char95APP}} (you \emph{do} write this, don'{}t you?) * {\colorbox[rgb]{0.87,0.87,0.87}{\tt test\char47test\char95helper\char46rb}}

Of course, these lists won'{}t apply in every situation, but in general I think that'{}s how it'{}ll break down. Now, on to the things you'{}ll need to change\ldots{}

\hypertarget{_is_dead_long_live_}{}\subsection*{{{\colorbox[rgb]{0.87,0.87,0.87}{\tt config\char46gem}} is dead, long live {\colorbox[rgb]{0.87,0.87,0.87}{\tt bundler}}}}\label{_is_dead_long_live_}

Everyone and their brother complained about Rails'{} handling of vendored/bundled gems since {\colorbox[rgb]{0.87,0.87,0.87}{\tt config\char46gem}} was added sometime ago (just search for ``{}config.gem sucks''{} or ``{}config.gem issues OR problems''{} and you'{}ll see). Between issues with requiring the gems properly to problems with the gem detection (I can'{}t tell you how many times I nixed a gem from the list because it kept telling me to install it even though it was already installed), Rails seriously needed a replacement for such a vital piece of infrastructure. These days we have \href{http://github.com/carlhuda/bundler}{Yehuda Katz'{}s excellent bundler}, which will be the standard way to do things in Rails 3.

Essentially, bundler works off of {\colorbox[rgb]{0.87,0.87,0.87}{\tt Gemfiles}} (kind of like {\colorbox[rgb]{0.87,0.87,0.87}{\tt Rakefiles}} in concept) that contain a description of what gems to get and how to get them. Moving your gem requirements to a {\colorbox[rgb]{0.87,0.87,0.87}{\tt Gemfile}} isn'{}t as simple as copying them over, but it'{}s not terribly difficult:

\begin{verbatim}# This gem requirement...
config.gem "aws-s3", :version => "0.5.1", 
           :lib => "aws/s3", :source => "http://gems.omgbloglol.com"

# ...becomes:
source "http://gems.omgbloglol.com"
gem "aws-s3", "0.5.1", :require_as => "aws/s3"\end{verbatim}
As you can see, it'{}s not too hard. It'{}s basically just removing the {\colorbox[rgb]{0.87,0.87,0.87}{\tt config}} object and moving some keys around. Here'{}s a specific list of changes:

* Remove the {\colorbox[rgb]{0.87,0.87,0.87}{\tt config}} object * {\colorbox[rgb]{0.87,0.87,0.87}{\tt \char58lib}} key becomes the {\colorbox[rgb]{0.87,0.87,0.87}{\tt \char58require\char95as}} key * The {\colorbox[rgb]{0.87,0.87,0.87}{\tt \char58version}} key becomes a second, optional string argument * Move {\colorbox[rgb]{0.87,0.87,0.87}{\tt \char58source}} arguments to a {\colorbox[rgb]{0.87,0.87,0.87}{\tt source}} call to add it to the sources

Once you create a {\colorbox[rgb]{0.87,0.87,0.87}{\tt Gemfile}}, you simply have to run {\colorbox[rgb]{0.87,0.87,0.87}{\tt bundle~pack}} and you'{}re done!

The {\colorbox[rgb]{0.87,0.87,0.87}{\tt bundler}} is much more powerful than {\colorbox[rgb]{0.87,0.87,0.87}{\tt config\char46gem}}, and it helps you do more advanced tasks (e.g., bundle directly from a Git repository, specify granular paths, etc.). So, once you move your {\colorbox[rgb]{0.87,0.87,0.87}{\tt config\char46gem}} calls over, you may want to look into the \href{http://github.com/carlhuda/bundler/blob/master/README.markdown}{new features}; they may be something you had wished {\colorbox[rgb]{0.87,0.87,0.87}{\tt config\char46gem}} had but didn'{}t!

\emph{Note: I'{}ve noticed some activity in Yehuda/Carl'{}s Githubs to do with a bundler replacement called gemfile; I'{}ll watch that closely to make sure there are no major breaking changes in the API/operation. If there are, I'{}ll definitely post here!}

\hypertarget{move_to_}{}\subsection*{{Move to {\colorbox[rgb]{0.87,0.87,0.87}{\tt Rails\char58\char58Application}}}}\label{move_to_}

In all previous Rails version, most configuration and initialization happened in {\colorbox[rgb]{0.87,0.87,0.87}{\tt config\char47environment\char46rb}}, but in Rails 3, most of this logic is moved to {\colorbox[rgb]{0.87,0.87,0.87}{\tt config\char47application\char46rb}} and a host of special initializers in {\colorbox[rgb]{0.87,0.87,0.87}{\tt config\char47initializers}}. The {\colorbox[rgb]{0.87,0.87,0.87}{\tt config\char47environment\char46rb}} file basically looks like this now:

\begin{verbatim}# Load the rails application
require File.expand_path('../application', __FILE__)

# Initialize the rails application
YourApp::Application.initialize!\end{verbatim}
Simple: the {\colorbox[rgb]{0.87,0.87,0.87}{\tt application\char46rb}} file is required and then the {\colorbox[rgb]{0.87,0.87,0.87}{\tt Application}} is initialized. The {\colorbox[rgb]{0.87,0.87,0.87}{\tt YourApp}} constant is generated based on the folder name for your app (i.e., {\colorbox[rgb]{0.87,0.87,0.87}{\tt rails~\char126\char47code\char47my\char95super\char95app}} would make it {\colorbox[rgb]{0.87,0.87,0.87}{\tt MySuperApp}}), so name it wisely! It doesn'{}t have any special relationship to the folder the app lives in so you can rename it at will (so long as you do it everywhere it'{}s used), but you'{}ll be using this constant in a few places so make it something useful.

Now you need an {\colorbox[rgb]{0.87,0.87,0.87}{\tt application\char46rb}}; if you generated the files using the Rails 3 generator, you should have one that looks something like this:

\begin{verbatim}module TestDevApp
  class Application < Rails::Application
    # ...Insert lots of example comments here...
    
    # Configure sensitive parameters which will be filtered from the log file.
    config.filter_parameters << :password
  end
end\end{verbatim}
For the most part, your {\colorbox[rgb]{0.87,0.87,0.87}{\tt config\char46\char42}} calls should transfer straight over: just copy and paste them inside the class body. There are a few new ones that I'{}ll be covering later on in this series that you might want to take advantage of. If you run into a {\colorbox[rgb]{0.87,0.87,0.87}{\tt config\char46\char42}} method that \emph{doesn'{}t} work (other than {\colorbox[rgb]{0.87,0.87,0.87}{\tt config\char46gem}} which obviously won'{}t work), then please post in the comments, and I'{}ll add it into a list here.

You'{}ll also notice that many things that were once in {\colorbox[rgb]{0.87,0.87,0.87}{\tt environment\char46rb}} have been moved out into new initializers (such as custom inflections). You'{}ll probably want to/have to move these things out of {\colorbox[rgb]{0.87,0.87,0.87}{\tt application\char46rb}} and into the proper initializer. If you opted to keep any custom initializers or specialized environment file during the generation process, you'{}ll probably need to go in there and update the syntax. Many of these (especially the environment files) now requires a new block syntax:

\begin{verbatim}# Rails 2.x
config.cache_classes = false
config.action_controller.perform_caching = true

# Rails 3.x
YourApp::Application.configure do
  config.cache_classes = false
  config.action_controller.perform_caching = true
end\end{verbatim}
All configuration happens inside the {\colorbox[rgb]{0.87,0.87,0.87}{\tt Application}} object for your Rails app, so these, too, need to be executed inside of it. As I said previously, most things in there should still work fine once wrapped in the block, but if they don'{}t please comment so I can post about it/figure out the issue.

\hypertarget{chchchaaange_in_the_router}{}\subsection*{{Ch-ch-chaaange in the router}}\label{chchchaaange_in_the_router}

You'{}ve probably heard a \href{http://yehudakatz.com/2009/12/26/the-rails-3-router-rack-it-up/}{lot of talk} about \href{http://rizwanreza.com/2009/12/20/revamped-routes-in-rails-3}{Rails and routes} and \href{http://github.com/jm/krauter}{new} \href{http://github.com/josh/rack-mount}{implementations} and this and that. Let me tell you: the new router is pretty awesome. The problem is that it'{}s not exactly easy to migrate existing routes over to the new hotness. Fortunately (for now, at least) they have a legacy route mapper so your routes won'{}t break any time soon. Of course, you should always try to update things like to this to keep up with the version you'{}re running (i.e., never depend on the benevolence of the maintainers to keep your ghetto legacy code going while using a new version for everything else).

But don'{}t worry. Upgrading your routes is fairly simple so long as you haven'{}t done anything complex; it'{}s just not as easy as copying and pasting. Here are a few quick run-throughs (a detailed guide is coming later)\ldots{}

Upgrading a basic route looks like this:

\begin{verbatim}# Old style
map.connect '/posts/mine', :controller => 'posts', :action => 'index'

# New style
match '/posts/mine', :to => 'posts#index'\end{verbatim}
A named route upgrade would look like:

\begin{verbatim}# Old style
map.login '/login', :controller => 'sessions', :action => 'new'

# New style
match '/login', :to => 'sessions#new', :as => 'login'\end{verbatim}
Upgrading a resource route looks like this:

\begin{verbatim}# Old style
map.resources :users, :member => {:ban => :post} do |users|
  users.resources :comments
end

# New style
resources :users do
  member do
    post :ban
  end

  resources :comments
end\end{verbatim}
And upgrading things like the root path and so on looks like this:

\begin{verbatim}# Old style
map.root :controller => 'home', :action => 'index'

map.connect ':controller/:action/:id.:format'
map.connect ':controller/:action/:id'

# New style
root :to => 'home#index'

match '/:controller(/:action(/:id))'\end{verbatim}
I'{}ll be writing another entry later on about the router'{}s new DSL and looking at some common patterns from Rails 2 apps and how they can work in Rails 3. Some of the new methods add some very interesting possibilities.

\hypertarget{some_minor_changes}{}\subsection*{{Some minor changes}}\label{some_minor_changes}

There are a few minor changes that shouldn'{}t really mess with too much (except perhaps the first one here\ldots{}).

\textbf{Constants are out, module methods are in} Ah, nostalgia. Remember when {\colorbox[rgb]{0.87,0.87,0.87}{\tt RAILS\char92\char95ROOT}} and friends were cool? Well, now they'{}re lame and are going away in a flare of fire and despair. The new sexy way to do it: {\colorbox[rgb]{0.87,0.87,0.87}{\tt Rails\char46root}} and \href{http://api.rubyonrails.org/classes/Rails.html}{its module method pals}. So, remember. Old and busted: {\colorbox[rgb]{0.87,0.87,0.87}{\tt RAILS\char92\char95ROOT}} and its depraved, constant brethren. New hotness: {\colorbox[rgb]{0.87,0.87,0.87}{\tt Rails\char46root}} and its ilk.

\textbf{Rack is Serious Business$^{\rm TM}${}} You might have noticed that the Rails 3 generator gives you a {\colorbox[rgb]{0.87,0.87,0.87}{\tt config\char46ru}} in your application root. Rails 3 is going gung ho on \href{http://rack.rubyforge.org/}{Rack}, everyone'{}s favorite web server interface, and as such, a {\colorbox[rgb]{0.87,0.87,0.87}{\tt config\char46ru}} is now required in your application to tell Rack how to mount it. Like I said, the Rails 3 generator will spit one out for you, but if you'{}re doing a manual upgrade for some reason, then you'{}ll need to add one yourself.

Interesting note: Remember that {\colorbox[rgb]{0.87,0.87,0.87}{\tt YourApp\char58\char58Application}} class you created earlier in {\colorbox[rgb]{0.87,0.87,0.87}{\tt application\char46rb}}? That'{}s your Rack endpoint; that'{}s why your {\colorbox[rgb]{0.87,0.87,0.87}{\tt config\char46ru}} looks like this:

\begin{verbatim}# This file is used by Rack-based servers to start the application.

require ::File.expand_path('../config/environment',  __FILE__)
run YourApp::Application.instance\end{verbatim}
Neat, eh? That'{}s why I suggested you pick a meaningful name for that class: its touch runs quite deep in the stack.

\textbf{.gitignore to the rescue} Rails also now automatically generates a {\colorbox[rgb]{0.87,0.87,0.87}{\tt \char46gitignore}} file for you (you can tell it not to by providing the {\colorbox[rgb]{0.87,0.87,0.87}{\tt \char45\char45skit\char45git}} option to the generator). It'{}s fairly simple, but covers the 95\% case for most Rails developers, and it'{}s certainly a welcome addition to the toolbox. It was always annoying having to create one every time and either dig up a previous one to copy or try to remember the syntax of how to make it ignore the same stuff.

After upgrading this stuff, you probably have a booting application. Of course, these are a lot of moving parts that could derail this plan: old and busted plugins, gem stupidity, weirdness in your config files, lib code that'{}s problematic, application code that needs upgrading (I can almost guarantee that), and so on. In any event, these are a few steps in the right direction; subsequent posts will show you the rest.

I apologize for not getting another \href{http://omgbloglol.com/post/344792822/the-path-to-rails-3-introduction}{Rails 3 upgrade post} up this weekend, but I spent this weekend working on a few things. First, I contributed a few little pieces to the Rails 3 release notes, which should be showing up on the Rails blog soon (\textbf{edit:} or view them \href{http://guides.rails.info/3_0_release_notes.html}{here} right now), but most of my time was devoted to a bigger project.

My little gem \href{http://github.com/jm/rails-upgrade}{{\colorbox[rgb]{0.87,0.87,0.87}{\tt rails\char45upgrade}}} is now \href{http://github.com/rails/rails\_upgrade}{{\colorbox[rgb]{0.87,0.87,0.87}{\tt rails\char92\char95upgrade}}}, an officially blessed upgrade tool that will be maintained by myself and the Rails team. You can get it from here: \href{http://github.com/rails/rails\_upgrade}{http://github.com/rails/rails\_upgrade}.

To use it now, simply install the plugin:

\begin{verbatim}script/plugin install git://github.com/rails/rails\_upgrade.git\end{verbatim}
The plugin adds the following tasks:

\begin{verbatim}rake rails:upgrade:check      # Runs a battery of checks on your Rails 2.x app
                              # and generates a report on required upgrades for Rails 3
rake rails:upgrade:gems       # Generates a Gemfile for your Rails 3 app out of your config.gem directives
rake rails:upgrade:routes     # Create a new, upgraded route file from your current routes.rb\end{verbatim}
Simply run those tasks in the same way you ran the commands with the {\colorbox[rgb]{0.87,0.87,0.87}{\tt rails\char45upgrade}} gem. In the near future, I plan on expanding the checks for deprecated pieces to handle some of the less obvious changes, adding some generators for other changes (like {\colorbox[rgb]{0.87,0.87,0.87}{\tt config\char47application\char46rb}}), and adding some extra tools (ideas/suggestions certainly welcome).

Anyhow, I'{}m really looking forward to seeing this project become a dependable upgrade tool. If you have any ideas or find any bugs, please contact me via e-mail or \href{http://twitter.com/jm}{Twitter} or, even better, fork it and handle it yourself!

Upgrading applications is good sport and all, but everyone knows that greenfielding is where the real fun is. At least, I love greenfielding stuff a lot more than dealing with old ghetto cruft that has 1,900 test failures (and 300 errors), 20,000 line controllers, and code that I'{}m pretty sure is actually a demon-brand of PHP.

Building a totally new app in Rails 3 is relatively simple (especially if you'{}ve done it in previous Rails versions), but there a few changes that can trip you up. In the interest of not missing a step someone may need, this post is a simple walkthrough of building a new app with Rails 3. I would have simply posted about the \href{http://guides.rails.info/getting_started.html}{Rails 3 version of the Getting Started guide}, but it'{}s actually a bit out of date now. I'{}ve committed each step in its own commit on Github so you can step through it (the repository is here: \href{http://github.com/jm/rails3_blog}{http://github.com/jm/rails3\_blog}).

\hypertarget{an_aside_installing_the_rails_3_beta}{}\subsection*{{An aside: Installing the Rails 3 beta}}\label{an_aside_installing_the_rails_3_beta}

Installing the Rails 3 beta can be sort of tricky since there are dependencies, it'{}s a prerelease gem, and \href{http://twitter.com/bitsweat/status/8662035599}{RubyGems basically poops the bed when those two scenarios collide}. Hopefully that'{}ll be fixed soon, but the mean time, install Rails'{} dependencies like so:

\begin{verbatim}gem install rails3b
gem install arel --pre

# or if that gives you hassle...

gem install i18n tzinfo builder memcache-client rack \
            rack-test rack-mount erubis mail text-format thor bundler\end{verbatim}
Once all those lovely gems are installed (add {\colorbox[rgb]{0.87,0.87,0.87}{\tt \char45\char45no\char45ri}} and {\colorbox[rgb]{0.87,0.87,0.87}{\tt \char45\char45no\char45rdoc}} if you want to skip those/speed up your install), then install the prerelease version of Rails:

\begin{verbatim}gem install rails --pre\end{verbatim}
Now you'{}re ready to roll on with the Rails beta!

\hypertarget{using_the_new_generator}{}\subsection*{{Using the new generator}}\label{using_the_new_generator}

The application generator is basically the same with two key differences:

\begin{itemize}%
\item The parameter that was formerly the app name is now the app path. You can still give it a ``{}name,''{} and it will create the folder like normal. But you can also give it a full path (e.g., {\colorbox[rgb]{0.87,0.87,0.87}{\tt \char126code\char47my\char95application}} rather than just {\colorbox[rgb]{0.87,0.87,0.87}{\tt my\char95application}}) and it will create the application there.
\item All parameters for the generator must go after the app path. So, previously one could do {\colorbox[rgb]{0.87,0.87,0.87}{\tt rails~\char45d~mysql~test\char95app}}, but now that has to be {\colorbox[rgb]{0.87,0.87,0.87}{\tt rails~test\char95app~\char45d~mysql}}. This change is largely due to the major refactoring of the Rails generators, so even though it'{}s somewhat of a temporary annoyance, it'{}s definitely worth it for the flexibility and power that the new generators bring (more on that soon).

\end{itemize}
So, let'{}s generate a blog application (really original, I know, right?):

\begin{verbatim}rails rails3_blog -d mysql\end{verbatim}
If you get an error like ``{}no value provided for required arguments `{}app\_path'{}``{}, then you'{}ve gotten your parameters out of order. If you'{}d like to use another database driver, you can provide {\colorbox[rgb]{0.87,0.87,0.87}{\tt postgresql}} or {\colorbox[rgb]{0.87,0.87,0.87}{\tt sqlite}} (or nothing, since {\colorbox[rgb]{0.87,0.87,0.87}{\tt sqlite}} is the default). You'{}ll see a lot of text scroll by, and now we have a nice, fresh Rails 3 application to play with.

Let'{}s crank up the server (note that it'{}s different now!)\ldots{}

\begin{verbatim}rails server\end{verbatim}
Rails went the ``{}Merb way''{} and has consolidated its many {\colorbox[rgb]{0.87,0.87,0.87}{\tt script\char47\char42}} commands into the {\colorbox[rgb]{0.87,0.87,0.87}{\tt rails}} binscript. So things like {\colorbox[rgb]{0.87,0.87,0.87}{\tt generate}}, {\colorbox[rgb]{0.87,0.87,0.87}{\tt server}}, {\colorbox[rgb]{0.87,0.87,0.87}{\tt plugin}}, etc. are now {\colorbox[rgb]{0.87,0.87,0.87}{\tt rails~generate}} and so on. Once the server'{}s booted, navigate over to {\colorbox[rgb]{0.87,0.87,0.87}{\tt http\char58\char47\char47localhost\char58\char51\char48\char48\char48}} and you should see a familiar friend:



Click on ``{}About your application'{}s environment''{} to see more information about the app you'{}ve generated.

\hypertarget{configuring_an_app}{}\subsection*{{Configuring an app}}\label{configuring_an_app}

Now comes the task of configuration. Again, not a whole ton of changes from previous versions, but navigating them can trip up the novice and journey(wo)man alike. First, setup all your database settings in {\colorbox[rgb]{0.87,0.87,0.87}{\tt database\char46yml}}; it'{}s just like previous versions of Rails, so no surprises there (and plenty of information abounds if you'{}re new to it).

Next, pop open {\colorbox[rgb]{0.87,0.87,0.87}{\tt config\char47application\char46rb}}. This is where much of the configuration information that once lived in {\colorbox[rgb]{0.87,0.87,0.87}{\tt config\char47environment\char46rb}} now lives. The portion you probably want to pay attention to most when making a new application is the block that defines your options for ORM, template engine, etc. Here'{}s the default:

\begin{verbatim}config.generators do |g|
  g.orm             :active_record
  g.template_engine :erb
  g.test_framework  :test_unit, :fixture => true
end\end{verbatim}
I'{}m going to stick with the defaults, but you could substitute in something like {\colorbox[rgb]{0.87,0.87,0.87}{\tt \char58datamapper}} or {\colorbox[rgb]{0.87,0.87,0.87}{\tt \char58sequel}} for {\colorbox[rgb]{0.87,0.87,0.87}{\tt \char58active\char95record}}, {\colorbox[rgb]{0.87,0.87,0.87}{\tt \char58haml}} for {\colorbox[rgb]{0.87,0.87,0.87}{\tt \char58erb}}, or {\colorbox[rgb]{0.87,0.87,0.87}{\tt \char58rspec}} for {\colorbox[rgb]{0.87,0.87,0.87}{\tt \char58test\char95unit}} (once they get it working with Rails 3). Doing so will set the generators for models, views, etc. to use your tool of choice (remember that whole technology agnosticism thing?); I don'{}t know if all these generators are available yet, but there are some available here.

The {\colorbox[rgb]{0.87,0.87,0.87}{\tt config\char47application\char46rb}} file also houses some configuration for other things.

\begin{itemize}%
\item If you need to configure internationalization, it'{}s been moved to {\colorbox[rgb]{0.87,0.87,0.87}{\tt application\char46rb}}. Rails 3 comes equipped with a really powerful i18n toolkit; if you haven'{}t seen it, you can learn a little more about it here. The defaults that Rails sets up will work for most people (default locale is {\colorbox[rgb]{0.87,0.87,0.87}{\tt en}} and all translations in the default directory are automatically imported), so you may not need to touch anything, but if you need to customize, this is the place to do it.
\item You may want to set a default timezone. I usually stick with UTC since it'{}s easy to convert on a per-user basis to their desired timezone, but you might want to set it your timezone or the server'{}s timezone.
\item Your favorite old haunts from {\colorbox[rgb]{0.87,0.87,0.87}{\tt config\char47environment\char46rb}} such as {\colorbox[rgb]{0.87,0.87,0.87}{\tt config\char46plugins}}, {\colorbox[rgb]{0.87,0.87,0.87}{\tt config\char46load\char95paths}}, etc. are still there (even though {\colorbox[rgb]{0.87,0.87,0.87}{\tt config\char46gems}} is not).

\end{itemize}
Other configuration bits like custom inflections, mime types, and so on have been moved out into their own initializers that you can find under {\colorbox[rgb]{0.87,0.87,0.87}{\tt config\char47initializers}}.

The last big piece of configuration you'{}ll need to add is a {\colorbox[rgb]{0.87,0.87,0.87}{\tt Gemfile}} for {\colorbox[rgb]{0.87,0.87,0.87}{\tt bundler}} (get more information on {\colorbox[rgb]{0.87,0.87,0.87}{\tt Gemfiles}} and {\colorbox[rgb]{0.87,0.87,0.87}{\tt bundler}} here and here). We already have a basic {\colorbox[rgb]{0.87,0.87,0.87}{\tt Gemfile}} that has the following:

\begin{verbatim}# Edit this Gemfile to bundle your application's dependencies.
source 'http://gemcutter.org'

gem "rails", "3.0.0.beta"

## Bundle edge rails:
# gem "rails", :git => "git://github.com/rails/rails.git"

gem "mysql"

## Bundle the gems you use:
# gem "bj"
# gem "hpricot", "0.6"
# gem "sqlite3-ruby", :require => "sqlite3"
# gem "aws-s3", :require => "aws/s3"

## Bundle gems used only in certain environments:
# gem "rspec", :group => :test
# group :test do
#   gem "webrat"
# end\end{verbatim}
Notice that it has added {\colorbox[rgb]{0.87,0.87,0.87}{\tt mysql}} as a dependency since that'{}s what we set as the database (or whatever driver you selected, for example, {\colorbox[rgb]{0.87,0.87,0.87}{\tt pg}} or {\colorbox[rgb]{0.87,0.87,0.87}{\tt sqlite}}). Since I want to write blog entries in Markdown, I'{}m going to add {\colorbox[rgb]{0.87,0.87,0.87}{\tt rdiscount}} as a dependency. To do so, I simply have to add this:

\begin{verbatim}gem "rdiscount"\end{verbatim}
As I'{}ve said before, {\colorbox[rgb]{0.87,0.87,0.87}{\tt bundler}} is much more powerful than {\colorbox[rgb]{0.87,0.87,0.87}{\tt config\char46gem}}, and one of the great features it adds is the concept of a gem ``{}group.''{} For example, let'{}s say I want to use {\colorbox[rgb]{0.87,0.87,0.87}{\tt mocha}}, but only when testing (obviously). You would add this to your {\colorbox[rgb]{0.87,0.87,0.87}{\tt Gemfile}}:

\begin{verbatim}group :test do
  gem "mocha"
end\end{verbatim}
Now this gem will only be added in when testing. This will also be useful for production only gems related to caching and what not.

Next, run {\colorbox[rgb]{0.87,0.87,0.87}{\tt bundle~pack}} if you want to vendor everything or {\colorbox[rgb]{0.87,0.87,0.87}{\tt bundle~install}} to install the gems to system gems. After you'{}ve combed through this stuff and set whatever you need, you'{}re done configuring your application. Now on to actually building something.


\end{document}
